\chapter{終わりに}\label{ux7d42ux308fux308aux306b}

最安定粒界エネルギーを求める過程において,これまで手動で行っていた構造最適化及び原子の削除を自動化することに成功した.
今後の課題として

    謝辞

本研究を進めるにあたり,多忙な中多大なるご指導ご鞭撻を頂きました西谷滋人教授に深く感謝いたします.
また,本研究における知識の供給と研究への協力をして頂いた西谷研究室の同輩,先輩の方々に感謝の意を表します.本当にありがとうございました.

    参考資料 {[}1{]}Trace only means that only the total pressure, i.e. the
line http://cms.mpi.univie.ac.at/vasp/guide/node112.html {[}2{]}岩佐恭佑
(2016).「原子削除操作を加えた対称傾角粒界のエネルギー計算」関西学院大学理工学部卒業論文(未公刊)
{[}3{]}William H.Press, Saul A. Teukolsky, William T. Vetterling,Brian
P.Flannery (1986) Numerical Recipes in C. The Art of Scientific
Computing Cambridge University Press
ウィリアム・H・プレス ソール・A・テウコルスキー ウィリアム・T・ベッターリング ブライアン・P・フランネリー 丹慶勝市・奥村晴彦・佐藤峻郎・小林誠(訳) (1993) 
ニューメカルレシピ・イン・シー 日本語版 C言語による数値計算のレシピ 技術評論社 pp.285-289
{[}4{]}5.7 POSCAR file
http://cms.mpi.univie.ac.at/vasp/guide/node59.html


    % Add a bibliography block to the postdoc
    
    
    
    \end{document}
