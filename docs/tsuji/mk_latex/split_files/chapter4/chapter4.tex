\chapter{終わりに}\label{ux7d42ux308fux308aux306b}

最安定粒界エネルギーを求める過程において,これまで手動で行っていた構造最適化及び原子の削除の自動化を試みた.構造最適化のための囲い込みをおこなうmnbrakを実装してみたが,現在行っている手動による指定よりも計算数を劇的に減らすことはできなかった.そこで,粒界近傍の原子削除の自動化を実装した.削除操作は,細かい指定が可能であり,得られたモデルの構造から最適化の最適値を見積もることが期待される.
このシステムを使用することで最安定の粒界エネルギーを求めることが容易になり,作業の効率化が期待できる.

今後の課題としては
\begin{itemize}
\item 構造最適化ではなく,他の操作を見直すことによるエネルギー計算回数の削減が出来るシステムの開発,
\item 自動原子削除において削除原子数を指定できる機能の追加,
\item 本研究では触れられなかった計算サーバへのファイル転送の自動化
\end{itemize}
が挙げられる.