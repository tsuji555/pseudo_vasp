\chapter{はじめに}\label{ux306fux3058ux3081ux306b}

    西谷研では最安定の粒界エネルギーを第一原理計算で求める研究を行っている.

この研究において,
第一原理計算はVASPという計算ソフトによって自動で行われるが,その前後の作業工程のいくつかが手動で行われている.
\begin{quote}
主な作業名称:コマンド名(自動化の度合い)
\end{quote}
をまとめると 
\begin{enumerate}
\item 原子モデル作成:modeler
\item 粒界セルモデル作成:make\_all(自動化済み)
\item 原子削除(手動)
\item 計算サーバへのファイル転送:scp(手動)
\item 計算設定ファイル:vasprun
\item ファイル配置(自動化済み)
\item 構造最適化の手動設定(手動)
\item 第一原理計算:vasp(自動)
\item 結果の解析:rake gets finishedn(自動化済み)
\end{enumerate}
である.作業全体の細かな手順は藤村がまとめている\cite{fujimura2018}.

これらは使い慣れた作業者に取っては,
間違った場合もすぐに気づくことができ,
間違いのケアも迅速に出来るという点では良い.
しかし,初心者がこれらの作業を手動でやると,
途中で何をしているのか分からなくなり,効率が悪くなってしまう

これらの手順の一部を自動で行ったり,
間違いを検出してくれるようなシステムを構築し,
初心者でも簡単に最安定な粒界エネルギーを求められるようにすることが本研究の目的である.
最初に構造最適化について記す.さらに,自動原子削除についての試みを記す.
    